\documentclass[ecp,tc,english]{inputs/iiufrgs}

\usepackage[utf8]{inputenc}   % pacote para acentuação

\usepackage{graphicx}           % pacote para importar figuras
\usepackage{svg}

\usepackage{times}              % pacote para usar fonte Adobe Times
\usepackage{verbatim}

\usepackage{pdfpages} % usado no anexo, para incluir arquivos PDF inteiros.
\usepackage{adjustbox} % usado numa tabela que ficou muito larga

\usepackage{csquotes}
\usepackage{hyperref}
\usepackage[hyperref=true,style=abnt,repeatfields=true,language=english]{biblatex}
\usepackage{float} 
\usepackage{amssymb}
\usepackage[ruled,vlined]{algorithm2e}
\usepackage[english]{babel}



\addbibresource{biblio.bib}

\title{Procedural generation of cave-like maps for 2D top-down games}
\author{Milani Rodrigues de Freitas}{Vinicius}
\advisor[Prof.~Dr.]{Couto Barone}{Dante Augusto}
\coadvisor[Dr.]{Batista Silva de Carvalho}{Leonardo Filipe}
\date{}{2021}
\location{Porto Alegre}{RS}

\makeatletter

\let\newtitle\@title
\let\newauthor\@author
\let\newdate\@date
\makeatother

\sloppy

\begin{document}

\setcounter{page}{1}
\thispagestyle{empty}
\phantomsection
\pdfbookmark[1]{Cover}{cover}
\begin{center}
\setstretch{1.0}{
UNIVERSIDADE FEDERAL DO RIO GRANDE DO SUL\\
INSTITUTO DE INFORMÁTICA\\
CURSO DE ENGENHARIA DE COMPUTAÇÃO
}
\end{center}

\vfill

\begin{center}
VINICIUS MILANI RODRIGUES DE FREITAS
\end{center}

\vfill

\noindent\parbox[t]{\textwidth}{%
\centering%
\vbox to 20mm{%
\parbox[b]{90mm}{%
\centering\vbox to 50mm{%
        {\large\textbf{\@\newtitle}\par}
        \vfill
}}%
}}

\vfill

\vfill

\begin{center}
\setstretch{1.0}{
Porto Alegre\\
2021
}
\end{center}



\keyword{video game}
\keyword{procedural generation}
\keyword{dungeon}
\keyword{Unity}
\keyword{cave}

\maketitle

% agradecimentos
\clearpage
\begin{flushright}
\mbox{}\vfill
{\sffamily\itshape
Dedico este trabalho à minha família, amigos e namorada.
}\end{flushright}


\chapter*{Acknowledgements}
texto de acknowledgement

\clearpage

% EPÍGRAFE
\begin{flushright}
\mbox{}\vfill
{\sffamily\itshape
``quote.''\\}
--- \textsc{quem fez quote}
\end{flushright}

% palavras-chave
% iniciar todas com letras minúsculas, exceto no caso de abreviaturas

% Keywords precisam estar definidos ANTES do \maketitle

\begin{abstract}
resumin
\end{abstract}

\begin{englishabstract}
{Geração procedural de mapas de cavernas para jogos 2D top-down}
{Jogo eletrônico, geração procedural, calabouço, Unity, caverna}
 resumim em br
\end{englishabstract}





% lista de figuras
\listoffigures

% lista de tabelas
\listoftables

% lista de abreviaturas e siglas
% o parametro deve ser a abreviatura mais longa

\begin{listofabbrv}{SPMD}
  \item[PCG] Procedural Content Generation
  \item[NPC] Non-Playable Character
  \item[AI] Artificial Intelligence
  \item[RPG] Role-Playing Games
  \item[CA] Cellular Automata
\end{listofabbrv}

% idem para a lista de símbolos
%\begin{listofsymbols}{$\alpha\beta\pi\omega$}
%       \item[$\sum{\frac{a}{b}}$] Somatório do produtório
%       \item[$\alpha\beta\pi\omega$] Fator de inconstância do resultado
%\end{listofsymbols}

% sumario
\tableofcontents



\chapter{Introduction} %fazer virar uma introduction
\label{chapter:intro}

Designing a game involves a conjunction of many different disciplines, for example: art, animation, sound, story, character creation, etc. Although the term game design is widely used in the industry to refer only to the design of the gameplay aspect, all the separate parts of design must properly work together in order to provide the player with a good experience \cite{zubek:2020}.

In this chapter we are going to present the necessary background information in order to understand what motivated this work. First, we will discuss the concept of game design and how it evolved through the ages. Then we will present some information about the game industry and the costs of developing a game. Lastly we will explain what procedural content generation (PCG) is and show a division of categories to better understand it.

After the motivation of the work is explained, we will present our objectives and preface the content of the next chapters. 

\section{Game design}

The importance of game design has increased throughout video game history. Designers of early games like Pong, shown here on Figure \ref{fig:pong}, or Spacewar! had very limited computing resources to work with, therefore it makes sense that they are simple and straightforward. Such games had little to no audio output, very primitive graphics and simple gameplay, sometimes borrowing ideas from well-established games, such as Pong, which is an electronic version of the Tabletennis sport \cite{wolf:2007}.

\begin{figure}[h]
    \caption{Atari's PONG arcade released in 1972}
    \centerline{\includegraphics{images/introduction/PongScreenAtariOriginal.jpeg}}
    \legend{Source: \cite{pongmuseum:2021}}
    \label{fig:pong}
\end{figure}

In time, the technological advances made possible for more complex games to be created. Designing a game became much more nuanced and time consuming. 

\subsection{Evolution of the video game industry}

In 2020, the video game industry was already bigger than the movie industry and North American sports combined. The gaming industry has also experienced a big growth thanks to the COVID-19 pandemic \cite{marketwatch:2019}, since players are having more time at home for their hobby. But even before the pandemic, the industry was already in fast-grow along the last decades, as shown in Figure \ref{fig:growth_graph}.

\begin{figure}[h]
    \caption{Video game industry revenue through the ages}
    \centerline{\includegraphics[width=13cm]{images/introduction/industry_growth.png}}
    \legend{Source: \cite{bloomberg:2019}}
    \label{fig:growth_graph}
\end{figure}

On the other hand, the cost of making games has increased dramatically as well. The cost of making Final Fantasy 7 remake, which was released in April 2020, for example, was roughly \$200 million, around \$120 million more than the original Final Fantasy 7, which was released in January 1997 \cite{cbr:2021}.

Triple-A games, which are games that are produced by mid-sized or major publishers \cite{steinberg:2007}, currently require the work of hundreds of people over the period of years to develop a single game. This is resulting in games not being as profitable for some developers, as few companies can afford developing long, diverse and polished games \cite{shaker:2016}.

One of the methods to reduce the cost of game designing is Procedural Content Generation (PCG), which uses algorithms to generate automatic content for the game.

\section{Procedural content generation}
\label{sec:pcg}

In the context of video games, PCG is defined as "the algorithmic creation of game content with
limited or indirect user input" \cite{togelius:2011}. In other words, PCG refers to software that is able to create game content by itself.

According to \textcite{doull:2008} there are 7 categories of PCG in games:

\begin{itemize}
\item \textbf{Runtime random level generation}: the generation of game levels while the game is being played. This is what people often think of when PCG in games is mentioned. In this category, an algorithm is responsible for generating random or pseudo-random levels for the game.

\item \textbf{Design of level content}: in this method, the automatically generated content is used at the level design stage to supplement human design skills. An algorithm can, for example, populate an environment created by the designer rapidly. Or the designer may choose specific generated levels to expand on.

\item \textbf{Dynamic world generation}: This technique is used to dynamically grow the environment that the player interacts on by using random seeds. In this case, the generated maps are never held in memory except as temporary structures to display.

\item \textbf{Instancing of in game entities}: in order to reach a statistically insignificant chance of repetition, in-game entities, such as like monsters, items, non-playable characters (NPCs), have some of their properties procedurally generated. These properties may be, for example, the position of the entity, its size, structure, etc.

\item \textbf{User mediated content}: this is a type of procedural generation where the user is in control. The technique offers a range of possibilities to users, who are responsible for putting them together in order to generate content.

\item \textbf{Dynamic systems}: some real-world systems such as group or weather behavior can be modelled using PCG techniques. This is widely used in combination of artificial intelligence (AI) in order to, for example, make NPCs react differently according to certain weather conditions.

\item \textbf{Procedural puzzles and plot generation}: this category is about using PCG in order to generate individual puzzle elements to increase replayability, e.g. changing door codes. Games that have its plot generated by PCG are also included in this category.
\end{itemize}

An example of a commercial game that was developed utilizing different PCG techniques is Electronic Arts's Spore. In this game, the player's objective is to evolve its own species, starting all the way from a microscopic organism until interstellar exploration. Entities on the game are generated by other players using in-game editors, which is an example of user mediated content generation. Worlds and galaxies are created through a combination of dynamic world generation and runtime random level generation. Moreover, some of the character animations are created procedurally \cite{wright:2007}. Figure \ref{fig:spore} shows user-generated characters standing on a procedurally generated world.

\begin{figure}[h]
    \caption{Screenshot of Spore gameplay}
    \centerline{\includegraphics[width=13cm]{images/introduction/spore.jpg}}
    \legend{Source: \cite{ea:2008}}
    \label{fig:spore}
\end{figure}


\section{Dungeons in games}

One of the most notable applications of PCG in games is the creation of procedurally generated dungeons. A dungeon is a labyrinthic environment that the player can explore, as well as collect items, slay monsters, fall into traps, etc. Although originally the term "dungeon" refers to a labyrinth of prison cells, nowadays, there are dungeons representing caverns, castles, forests, underwater environments, etc \cite{shaker:2016}. 

This current definition of a dungeon can be probably tracked to the Dungeons and Dragons game, which is a tabletop Role-Playing Game (RPG) that had a huge influence in the development of video games throughout history. In fact, a specific type of computer RPG has spawned from the idea of exploring randomly generated dungeons: the roguelike genre. The name "roguelike" comes from the 1980's game Rogue, which featured procedurally generated dungeons that the player had to explore in search of an amulet \cite{brewer:2016}. A screenshot of a typical Rogue session can be seen in Figure \ref{fig:rogue}.

\begin{figure}[h]
    \caption{Screenshot of Rogue gameplay}
    \centerline{\includegraphics[width=13cm]{images/introduction/rogue.jpg}}
    \legend{Source: \cite{epyx:1985}}
    \label{fig:rogue}
\end{figure}

In \textcite{melan:2006} it is suggested that the design of the dungeon structure is very important to the creation of a good dungeon map. According to the author, a good map design is one that embodies the factors that make playing on a dungeon fun: exploration, decision making, consistent pace of action, discovery of secrets. Four different basic forms of dungeon maps, created from the author experience with RPG, were presented in \textcite{melan:2006},  shown here on Figure \ref{fig:basic_dungeons}. 

\begin{figure}[h]
    \caption{Basic dungeon structures}
    \centerline{\includegraphics[width=13cm]{images/introduction/basic_dungeon.png}}
    \legend{Source: \cite{melan:2006}}
    \label{fig:basic_dungeons}
\end{figure}



\section{Objectives}
\label{sec:objectives}

Taking motivation from the information presented in the previous sections, this work has been elaborated with the objective of creating a PCG system to design cave-like dungeon maps for 2D top-down games.

More specifically, we will:

\begin{itemize}
\item \textbf{} Utilize a number of algorithms that, combined with user inputs, can generate varied and cave-like dungeon maps.

\item \textbf{} Focus on the PCG category of 'design of level content'. The generated levels can serve as a blueprint for game designers to build upon on the future.

\item \textbf{} Create the maps based on a set of criteria found in the bibliography for what makes a good dungeon map, including making maps that can be categorized in one or more of the basic forms of Figure \ref{fig:basic_dungeons}.

\item \textbf{} Evaluate if the generated maps match the criteria with the application of a survey.
\end{itemize}

\section{Organization of this work}

On Chapter \ref{chapter:related} we will cover works that are related to this one. After that, on Chapter \ref{chapter:proposal} some important information about our proposal is presented. Then, on chapter \ref{chapter:dev} we will provide details on the development and implementation. Chapter \ref{chapter:survey} will cover the creation of the survey and its results. And, lastly on Chapter \ref{chapter:conclusion} we will conclude this work.


\chapter{Related Work}
\label{chapter:related}

While looking for works related to this one, we searched for works that focused on dungeon or content generation specifically for games and also for works that had their focus on cave generation.

\section{Conditional Convolutional Generative
Adversarial Networks Based Interactive
Procedural Game Map Generation}

This paper by \citeauthor{ping:2020} suggests the usage of Conditional Generative Adversarial Network and Convolutional Neural Network to create a game design system. The system takes by input a gameplay area map defined by the user and generates a complex map with the same design pattern.

\begin{figure}[h]
    \caption{Example of the system's usage}
    \centerline{\includegraphics{images/related_work/ping.png}}
    \legend{Source: \cite{ping:2020}}
    \label{fig:ping}
\end{figure}
In Figure \ref{fig:ping} we see in the first two steps the image designed by the user and in the last step the generated map. The variety of the maps depend on the training of the network as well \citeyear{ping:2020}.

\section{Cellular automata for real-time generation
of infinite cave levels}

In this paper a simple cellular automata (CA) algorithm is used in order to generate, in real time, infinite cave-like levels. The algorithm was tested in a game called Cave Crawler, which is a birds-eye view game where the players have to traverse the generated tunnels while defeating waves of enemies. 

Much like this paper, our work also utilizes a simple CA algorithm as one of the steps of the system in order to generate the cave levels, but on this paper the focus is more on the real-time and performance aspect of the map generation \cite{johnson:2010}.

\section{Procedural creation of 3D solution cave models}

The focus of this article by \citeauthor{boggus:2009} is to use PCG algorithms to create 3D cave models based on real cave patterns. They present the usage of surface images of cave patterns and apply these images to create models of cave systems. The surface images can be generated, for example by utilizing fractal algorithms, or real world terrain data.

After the surface image is provided, they create a heightmap out of it and then simulate the flow of water through this map, since this is ultimately the way solution caves are formed. Figure \ref{fig:3d_cave} shows a cross section view of a generated cave, where lighter areas represent walls \citeyear{boggus:2009}.
\begin{figure}[h]
    \caption{Spongework cave pattern}
    \centerline{\includegraphics[width=7cm]{images/related_work/3d_cave.png}}
    \legend{Source: \cite{boggus:2009}}
    \label{fig:3d_cave}
\end{figure}

\section{Procedural Playable Cave Systems
based on Voronoi Diagram and Delaunay Triangulation}

This paper by \citeauthor{santamaria:2014} presents a new method for 2D or 3D playable cave systems to be generated. The user-defined input for their method are a list of points of interest (POI) that will be put inside the cave and the relationship between these POI. The paper defines a list of parameters that compose the POI like location, depth, number of branches etc.

\begin{figure}[h]
    \caption{Cave generation process}
    \centerline{\includegraphics[width=6cm]{images/related_work/poi_cave.png}}
    \legend{Source: \cite{santamaria:2014}}
    \label{fig:poi_cave}
\end{figure}

Figure \ref{fig:poi_cave} shows their cave generation process. In (a) a Voronoi diagram is generated on a terrain representation. (b) shows the classification of the cells. (c) shows 7 POI assigned to their respective cells in the diagram and (d) shows the final generated cave \citeyear{santamaria:2014}.

\section{Analysis and Development of a
Game of the Roguelike Genre}

This work describe the development of a prototype game of the roguelike genre, and presents an analysis focused on the different techniques used on the AI for enemies and on the procedural generation of the dungeons. The compared PCG algorithms include: Kruskal and Prim, both being genetic algorithms; depth-first search, used to generate perfect mazes and cellular automata for cave-like structures. Some of the gathered results show that techniques of basic interaction iteration and BSP trees have good scalability while cellular automata and depth-first search need optimization or restriction in other to increase scalability \cite{goncalves:2015}.

\chapter{Proposal}

In this chapter we will detail our proposed solution that was developed in order to achieve the objectives presented in the \nameref{sec:objectives} section. We will define some concepts that will be necessary in order to fully understand the next chapter, where the development process and difficulties found of the proposed solution will be described.


pegar do mestrado do leaonardo definição de mapa coisassin

falar de unity

falar das caverna

por fim falar da proposta: gerar um bom mapa em unity, o que é um bom mapa,
\chapter{Development}
\label{chapter:dev}

On this chapter we will be explaining how the system was implemented, which algorithms are used and why, and what were the challenges during development.

The system was developed in the Unity engine, utilizing the C\# language. The base of the work was started by following the video tutorial found here \cite{taft:2019}, about creating a 2D top-down game in Unity. Here is where some of the foundation for the work was made, like importing the tileset, understanding how to draw a map, basic collision etc.

After this point is where the creation of the PCG system started. To better understand its implementation we will divide it in 7 steps and discuss each one in a different Section of this Chapter, the last Section will cover all of the user inputs that can be defined in the system in order to tweak the map generation. But first, some information that will be useful to understand all the steps from this point forward: 
\begin{itemize}
    \item The generated map will be represented by a global matrix of integers with width \(w_{map}\) and height \(h_{map}\).
    \begin{itemize}
        \item Its important to notice that \(w_{map}\) and \(h_{map}\) are user-defined parameters.
    \end{itemize}
    \item Each individual integer of this matrix will be referred to as a \emph{cell}.
    \item A cell of value \(1\) represents a \emph{wall}, a cell of value \(0\) represents the absence of a \emph{wall}, or a path.
    \item The Figures where the map is shown are created by the use of Unity's Gizmo debugging tool. On these figures, white represents a \emph{path} cell and black represents a \emph{wall} cell.
\end{itemize}

\section{Create a randomly-filled base map}

The first step taken was to have a randomly generated base map to start building upon.

The generation of random or pseudo-random content generation is a part of most PCG techniques. In our case, the importance of the randomness is given by one of the items for what makes a good map, presented on Section \ref{sec:goodmap} of Chapter \ref{chapter:proposal}: "the Generated maps should look different from one another".

The creation of this base map is accomplished by the steps represented here in Algorithm \ref{alg:randomgen}.


\begin{algorithm}[h] 
 \DontPrintSemicolon
 int map[\(w_{map},h_{map}\)]\;
 \SetKwFunction{FillMapRandomly}{FillMapRandomly}
 \SetKwProg{Fn}{Function}{:}{}
 \Fn{\FillMapRandomly{}}{
  \For{all cells in the map}{
   random = randomly generated number between 0 and 100\;
   \eIf{random < fillPercent}{
    map cell = 1\;
   }{
    map cell = 0\;
   }
  }
 }
 \caption{Randomly filling the map}\label{alg:randomgen}
\end{algorithm} 

The variable \(fillPercent\) will determine how much of the map will be populated by \emph{walls} or \emph{paths}, this variable is a user-defined parameter, and different values will generate different maps at the end. Figure \ref{fig:fillper50} shows an example of a base map with \(w_{map} = h_{map} = 50\) and  \(fillPercent = 50\), and Figure \ref{fig:fillper80} shows the same map but with \(fillPercent = 80\).

\begin{figure}[h]
  \centering
  \begin{minipage}[b]{0.4\textwidth}
    \caption{Map with fillPercent = 50.}
    \includegraphics[width=\textwidth]{images/development/50_filled.png}
    \legend{Source: Image provided by author}
    \label{fig:fillper50}
  \end{minipage}
  \hfill
  \begin{minipage}[b]{0.4\textwidth}
    \caption{Map with fillPercent = 80.}
    \includegraphics[width=\textwidth]{images/development/80_filled.png}
    \legend{Source: Image provided by author}
    \label{fig:fillper80}
  \end{minipage}
\end{figure}

The random number generator used can also be controlled by an user-defined parameter. The user can choose to write a \emph{seed} for the generator, in this case the generated map will always be the same for each given \emph{seed}. Otherwise, the \emph{seed} will be chosen automatically based on the execution time, making each generation different from the previous.

\section{Cellular automata}

A CA is a discrete model of computation first discovered by Stanislaw Ulam and John von Neumann; it can be described, in a simplified manner, as an \(n\)-dimmensional grid, a set of states and a set of transition rules. The grid is composed by cells, and these cells can be in one of several states; in the most basic form, cells can be 1 (on) or 0 (off) \cite{wolfram:1983}.

Although CA has found application in many different areas of study, it was the 1970s Conway's Game of Life which brought the it's attention to beyond academia. This was a zero person game where cells from a 2D grid would live or die (only two states, \(1\) or \(0\)) based on how many neighbors they had around them. This neighborhood is defined as the 8 cells surrounding the central cell currently being analyzed. Some of the rules from the Game of Life have been applied in dungeon generation, because of the cave-like appearance that the algorithm creates on the grid \cite{shaker:2016}.

In our specific case, the rule used is that, if there are more than \(4\) \emph{walls} around the current cell, then the cell becomes a \emph{wall}, if there are less then \(4\) \emph{walls} around the cell, it becomes a \emph{path}. The method was also chosen because it generates images that are similar to the representation of \emph{spongework} caves, shown in Figure \ref{fig:cave_patterns}. Figure \ref{fig:ca_rule} serves as a visual representation to this rule, where the red cell is the one being currently analyzed and the number on the right represents is resulting state after the CA pass.

\begin{figure}[h]
    \caption{Representation of the CA rule used.}
    \centerline{\includegraphics[width=4cm]{images/development/ca_rule.png}}
    \legend{Source: Image provided by author}
    \label{fig:ca_rule}
\end{figure}

In Figure \ref{fig:5itca} we can see the base map generated in Figure \ref{fig:fillper50} after \(5\) iterations of the CA. And Figure \ref{fig:20itca} shows the map after \(20\) iterations. The number of iterations is also a user-defined parameter for the system.

\begin{figure}[h]
  \centering
  \begin{minipage}[b]{0.4\textwidth}
    \caption{Map after 5 iterations of CA.}
    \includegraphics[width=\textwidth]{images/development/5it_ca.png}
    \legend{Source: Image provided by author}
    \label{fig:5itca}
  \end{minipage}
  \hfill
  \begin{minipage}[b]{0.4\textwidth}
    \caption{Map after 20 iterations of CA.}
    \includegraphics[width=\textwidth]{images/development/20it_ca.png}
    \legend{Source: Image provided by author}
    \label{fig:20itca}
  \end{minipage}
\end{figure}

\section{Create a list of rooms}

At this stage, we are left with one or more rooms scattered throughout the map. We can define room by one or more connected \emph{path} cells. On code, we represent the rooms as a separate \emph{Room} class, that contains the coordinates of all cells on the room. It's important to note that the \emph{Room} class also contains some other important information are used by the step that will be explained in Section \ref{sec:connection}: a flag saying if the \emph{Room} was visited, a list of \emph{Rooms} connected to this \emph{Room}, a method that connects two \emph{Rooms}.

There are two sets of user-defined inputs that will determine: if there will be entrance and exit rooms created; and their location on the map. After the creation of these rooms, we go through the map grid and search for paths, if a path was found we use the flood-fill algorithm in order to define where this room begins and ends, and then create a \emph{Room} containing all the coordinates of its cells. After the creation of the \emph{Room}, we add it to a list, containing all \emph{Rooms} on the map.

\section{Checking if the generated map is possible}
\label{sec:check_poss}

One of the issues that came up during development is that not all generated maps can translate well to a 2D top-down tile-based map. This happens because, for a visually well-formed wall to be generated, there needs to have at least \(2\) \emph{walls} between \(2\) \emph{paths}. Figure \ref{fig:malformed_wall} exemplifies a malformed wall between 2 \emph{Rooms}.

\begin{figure}[h]
    \caption{Example of a malformed wall between two \emph{Rooms}}
    \centerline{\includegraphics[width=10cm]{images/development/deformed_wall.png}}
    \legend{Source: Image provided by author}
    \label{fig:malformed_wall}
\end{figure}

In this step we go through each cell of each \emph{Room} and check if there are \emph{Rooms} that have a distance of less then \(2\) cells from each other. If there are, then the whole process is started again until the generated \emph{Room} can be translated well to a tile-based map. Algorithm \ref{alg:check_possible} describes this process.

\begin{algorithm}[h]
 \DontPrintSemicolon
 \While{map is not possible}{
  Randomly fill the map\;
  Run cellular automata\;
  Create a list of Rooms\;
  Check if the generated map is possible\;
 }
 \caption{Checking if the generated map is possible}
\label{alg:check_possible}
\end{algorithm} 

\section{Connection between rooms}
\label{sec:connection}

We will divide this section into 2 steps: ensuring connectivity through graph algorithms and drawing the connection using Bresenham's line algorithm.

\subsection{Ensuring connectivity}

Ensuring the connectivity between all rooms is important for a reason cited on Section \ref{sec:goodmap} of Chapter \ref{chapter:proposal}: "Dungeon maps should have an entrance, an exit (sometimes being the same), and a path between them". For our specific case, there is one more reason why this step is important: making maps that are similar to the \emph{branchwork} or \emph{network} cave patterns.

We need to make sure that every pair of \emph{Rooms} are connected, i.e., there is a way to reach one \emph{Room} by starting on the other \emph{Room}. All of \emph{Rooms} in the map can be viewed as nodes in a graph, that way, we made use of a popular algorithm in graph theory to check if there's connectivity: depth-first search.

Figure \ref{fig:connectivity} shows a visual representation of Algorithm  \ref{alg:connectivity} used in this step. The nodes of the graph represent \emph{Rooms} in the map. In quadrant \(1\) we start with a map with fully disconnected \emph{Rooms}. We begin by connecting each node to its closest node in quadrant \(2\); then we do a depth-first search to check if connectivity is ensured; if it's not, we treat the sets of connected nodes as one node like is shown on quadrant \(3\), and then redo the connection step, the result can be seen in quadrant \(4\).

\begin{figure}[h]
    \caption{Visual representation of the connectivity algorithm.}
    \centerline{\includegraphics[width=7cm]{images/development/graph_connect.png}}
    \legend{Source: Image provided by author}
    \label{fig:connectivity}
\end{figure}

\begin{algorithm}[h]
 \DontPrintSemicolon
 Graph = represent Rooms as graph\;
 \While{connectivity isn't reached}{
  Connect closest Rooms in Graph\;
  Check connectivity\;
  Transform the sets of connected Rooms into new nodes for the new Graph\;
 }
 \caption{Ensuring connectivity of the map}
\label{alg:connectivity}
\end{algorithm} 

\subsection{Drawing the connection}

At this point we have a map filled with \emph{Rooms} that have ensured connectivity, but this connectivity is only represented by some elements of the \emph{Room} class, it is not yet represented in the map grid as \emph{walls} and \emph{paths}.

This is where the algorithm known as Bresenham's line is used. This algorithm, suggested in 1962 by Jack Elton Bresenham, is used to determine which points on an \(n\)-dimensional raster should be selected in order to form a close approximation of a straight line \cite{bresenham:1965}. 

Since each \emph{Room} has all of its cell coordinates stored, we can choose the closest cells for each connection and then use the Bresenham's line algorithm to draw this line on the map. Figure \ref{fig:map_connected} shows the same map from Figure \ref{fig:20itca} after the connectivity is ensured and then drawn on the map.

It may seem like these passages will not allow the player to pass through them, but the nature of the chosen tileset allows this passage to happen, as shown in Figure \ref{fig:it_works}.

\begin{figure}[h]
    \caption{Map after connectivity is ensured.}
    \centerline{\includegraphics[width=7cm]{images/development/after_connection.png}}
    \legend{Source: Image provided by author}
    \label{fig:map_connected}
\end{figure}

\begin{figure}[h]
    \caption{Passage before and after tile placement.}
    \centerline{\includegraphics[width=7cm]{images/development/it_works.png}}
    \legend{Source: Image provided by author}
    \label{fig:it_works}
\end{figure}

\section{Remove inconsistencies from the map}

helo guys.

Just like in Section \ref{sec:check_poss}, maps that are generated with single-spaced \emph{wall} will not translate well to a 2D tile-map. Some of these problematic \emph{walls} sometimes are created on the process of drawing the Bresenham's lines connecting two \emph{Rooms}.

In this step we deal with these here called \emph{inconsistencies} by going through the entire map grid and checking for \emph{wall} cells that have either a \emph{path} to its right and left, here shown on Figure \ref{fig:hor_inc}; or to its up or down, as shown in Figure \ref{fig:ver_inc}; the inconsistencies are circled in red.

\begin{figure}[h]
  \centering
  \begin{minipage}[b]{0.4\textwidth}
    \caption{Example of a horizontal inconsistency.}
    \includegraphics[width=\textwidth]{images/development/horizontal_incosistency.png}
    \legend{Source: Image provided by author}
    \label{fig:hor_inc}
  \end{minipage}
  \hfill
  \begin{minipage}[b]{0.4\textwidth}
    \caption{Example of a vertical inconsistency.}
    \includegraphics[width=\textwidth]{images/development/vertical_incosistency.png}
    \legend{Source: Image provided by author}
    \label{fig:ver_inc}
  \end{minipage}
\end{figure}

\section{Tile placement}

Finally, the last step of the generation is to place the tiles to their corresponding spots in the \emph{walls} and \emph{paths}.

There are a number of \(13\) \emph{wall} tiles needed to create the map, are shown here on Figure \ref{fig:wall_tiles}. There is only a single \emph{path} tile, which is visible on the resulted map in Figure \ref{fig:map_result}.

\begin{figure}[h]
    \caption{Wall tiles.}
    \centerline{\includegraphics{images/development/wall_tiles.png}}
    \legend{Source: Image provided by author}
    \label{fig:wall_tiles}
\end{figure}

The first idea we had was to try to use a custom CA ruleset, where each cell had \(14\) possible states, one for each needed tile. But, since after only one passage of the ruleset through the map grid all the tiles were correctly placed, this algorithm can be better described as a morphological operation.

This ruleset, containing 14 rules, is here shown on Figure \ref{fig:ruleset}. The next item list serves to better understand the content of this Figure:

\begin{itemize}
    \item Similarly to Figure \ref{fig:ca_rule}, the center cell is the one currently being analyzed.
    \item Gray cells represent \emph{wall} cells.
    \item White cells represent \emph{path} cells.
    \item Pink cells represent either a \emph{wall} or a \emph{path}, meaning that the content of this cell doesn't matter, i.e., won't be analyzed by this rule.
    \item The tile after the rule number is what the rule will return for the cell being analyzed. 
\end{itemize}

For example, \emph{Rule 3} will check if the current cell is a \emph{wall}, then if the cell above the current is a \emph{path}, then if the cell to the left is a \emph{path}, then if the cell to the right is a \emph{wall}, then if the cell below is a \emph{wall} and finally if the cell to the bottom right is a \emph{wall}; if all of these conditions are met, the rule will return the corner tile shown on Figure \ref{fig:ruleset} for \emph{Rule 3} and place it on the analyzed cell.

\begin{figure}[h]
    \caption{Ruleset.}
    \centerline{\includegraphics[width=12cm]{images/development/morph_rules.png}}
    \legend{Source: Image provided by author}
    \label{fig:ruleset}
\end{figure}

After the algorithm has gone through the whole map grid, the result will be a fully-formed 2D tile-based top-down map. Figure \ref{fig:map_result} shows the same map from Figure \ref{fig:map_connected} after the tile placement.

\begin{figure}[h]
    \caption{Resulting generated map.}
    \centerline{\includegraphics{images/development/final_result.png}}
    \legend{Source: Image provided by author}
    \label{fig:map_result}
\end{figure}

\section{User-defined parameters}

On this section we will be showcasing all of the user-defined parameters that can be used to tweak the generation of the map.

Figure \ref{fig:user_par} shows a snippet of the Unity tool, showing all the parameters used to generate the map shown on Figure \ref{fig:map_result}: \(w_{map} = h_{map} = 50\); the creation of a start and an end room was disabled; the number of iterations for the CA step was set to \(20\); a specific \emph{seed} was set; and the fill percent of the first step was set to \(50\). 

\begin{figure}[h]
    \caption{Available user parameters.}
    \centerline{\includegraphics{images/development/user_parameters.png}}
    \legend{Source: Image provided by author}
    \label{fig:user_par}
\end{figure}
\chapter{Survey}

falar pesquisa, mostrar resultados, os caraio
%\chapter{Survey}

falar pesquisa, mostrar resultados, os caraio
\chapter{Conclusion}
\label{chapter:conclusion}

This work showed how PCG is important in game design and how the price of game development has been increasing throughout the years. To aid in that matter, it proposed the development of a PCG system to generate random cave-like maps that are similar to real-world caves. Finally it evaluated this system through a survey.

One of the biggest challenges faced was the research for criteria on what makes a good map, that can be seen on Section \ref{sec:goodmap}. Most of the related work we found focused on video-game \emph{levels}, which also contain additional elements like enemies and treasures. Due to this difficulty we decided to also adopt the metrics found in \textcite{carvalho:2016} to evaluate players' motivation to explore game levels in educational games, which, in turn, is an adaptation of the ARCS model measurement tool found in \textcite{keller:1987}. To meet our ends, the tool from \textcite{carvalho:2016} was adapted to verify players' perception on our auto-generated caves, which demanded a pre-test of the questionnaire to verify if it indeed attended our needs. In this sense, the evaluation method for generated maps proposed on this work can be further reused by others with little to no adaptation.

The creation of the system required knowledge from different areas and the use and adaptation of many algorithms. Our system has many user-defined parameters, which is a positive trait for map generation, since it allows for a larger variety of maps to be created. Even though the work focused on cave-like maps, by changing the used tiles the system can generate maps like forests, open fields, rivers, etc. Figure \ref{fig:river} shows a procedurally generated river, also created with the use of the system.

\begin{figure}[h]
    \caption{River generated by the system.}
    \centerline{\includegraphics[width=6cm]{images/river.png}}
    \legend{Source: Image provided by author}
    \label{fig:river}
\end{figure}

The survey was answered by 163 participants, most of them from the age group that composes the average gamer and familiar to 2D top-down games. From the 12 questions, 9 achieved satisfactory results while 3 did not. Out of these results we found that, even though the structure of the maps resembles natural caves, the maps were not found to be much similar to 2D representations of real caves.

Finally, as a future works, the system could be turned into an Unity tool, that can then be used to create full games. The generation of rivers can be expanded on, by comparing them to real-life rivers and evaluating with a similar survey. The generation of maps can be changed to a level-generator, where the system wouldn't only generate the map but also place enemies, treasures, hidden passages etc. A deeper analysis of the questions regarding the ARCS model can be done to better evaluate the motivation to explore given to players by our maps.

\selectlanguage{english}
\printbibliography

l;\appendix

\chapter{Map figures used in both versions of the questionnaire}
\label{appendix:a}

\begin{figure}[h]
    \caption{Maps used in the questionnaires}
    \centerline{\includegraphics[width=10.505cm]{images/survey/allmaps.png}}
    \legend{Source: Image provided by author}
\end{figure}

\chapter{Pretest questionnaire}
\label{appendix:b}

\begin{table}[h]
\caption{Part of the pretest questionnaire regarding the maps.}
\resizebox{\textwidth}{!}{%
\begin{tabular}{|l|l|l|}
\hline
No. & Question & Justification \\ \hline
1 & \begin{tabular}[c]{@{}l@{}}There was something interesting in the cave \\ maps that caught my attention.\end{tabular} & \begin{tabular}[c]{@{}l@{}}Question related to the Attention component of the \\ ARCS model.\end{tabular} \\ \hline
2 & \begin{tabular}[c]{@{}l@{}}The first time I saw the cave maps, I had the \\ impression that they would be easy to explore.\end{tabular} & \begin{tabular}[c]{@{}l@{}}Question related to the Confidence component of \\ the ARCS model.\end{tabular} \\ \hline
3 & \begin{tabular}[c]{@{}l@{}}The structure of the caves was harder to \\ comprehend than I would have liked them to be.\end{tabular} & \begin{tabular}[c]{@{}l@{}}Question related to the Confidence component of\\ the ARCS model.\end{tabular} \\ \hline
4 & \begin{tabular}[c]{@{}l@{}}While looking at the maps, I feel that it would\\  be easy for me to get lost.\end{tabular} & \begin{tabular}[c]{@{}l@{}}Question related to the criteria: maps should have\\ \\ an entrance, an exit, and a path between them.\end{tabular} \\ \hline
5 & \begin{tabular}[c]{@{}l@{}}The structure of the map looked natural, it didn't \\ look like it was generated by an algorithm.\end{tabular} & \begin{tabular}[c]{@{}l@{}}Question related to the criteria: generated maps\\ should have a natural look.\end{tabular} \\ \hline
6 & \begin{tabular}[c]{@{}l@{}}While I observed the maps, I felt like I was\\ seeing a representation of real caves.\end{tabular} & \begin{tabular}[c]{@{}l@{}}Question related to the criteria: generated maps\\ should have a natural look.\end{tabular} \\ \hline
7 & \begin{tabular}[c]{@{}l@{}}In the generated maps, what reminds me of a \\ real cave are the used colors, and not the \\ structure of paths.\end{tabular} & \begin{tabular}[c]{@{}l@{}}This question serves to evaluate how important\\ the structure is in the creation of a natural look.\end{tabular} \\ \hline
8 & \begin{tabular}[c]{@{}l@{}}In the generated maps, what reminds me of a\\ real cave is the structure of paths, and not the\\ used colors.\end{tabular} & \begin{tabular}[c]{@{}l@{}}This question serves to evaluate how important\\ the used colors are in the creation of a natural look.\end{tabular} \\ \hline
9 & \begin{tabular}[c]{@{}l@{}}The generated map's design made it difficult \\ for me to keep my attention.\end{tabular} & \begin{tabular}[c]{@{}l@{}}Question related to the Attention component of \\ the ARCS model.\end{tabular} \\ \hline
10 & \begin{tabular}[c]{@{}l@{}}The generated cave maps were capable of\\ capturing my attention.\end{tabular} & \begin{tabular}[c]{@{}l@{}}Question related to the Attention component of\\ the ARCS model.\end{tabular} \\ \hline
11 & \begin{tabular}[c]{@{}l@{}}While looking at the maps, I feel like exploring\\ them.\end{tabular} & \begin{tabular}[c]{@{}l@{}}Question related to the criteria: maps should\\ encourage players to explore.\end{tabular} \\ \hline
12 & \begin{tabular}[c]{@{}l@{}}The design of the generated cave maps is \\ attractive.\end{tabular} & \begin{tabular}[c]{@{}l@{}}This question serves to evaluate the overall\\ attractiveness of the maps.\end{tabular} \\ \hline
13 & \begin{tabular}[c]{@{}l@{}}The design of the maps is very simple and not\\ attractive.\end{tabular} & \begin{tabular}[c]{@{}l@{}}This question serves to evaluate the overall \\ attractiveness of the design.\end{tabular} \\ \hline
14 & \begin{tabular}[c]{@{}l@{}}The generated structures look very similar,\\ with little variety.\end{tabular} & \begin{tabular}[c]{@{}l@{}}Question related to the criteria: maps should\\ look different from one another\end{tabular} \\ \hline
15 & \begin{tabular}[c]{@{}l@{}}The variety of the maps helps to keep my\\ attention.\end{tabular} & \begin{tabular}[c]{@{}l@{}}Question related to the criteria: maps should\\ look different from one another.\end{tabular} \\ \hline
\end{tabular}%
}
\legend{Source: Table provided by author.}
\end{table}

\begin{table}[H]
\caption{Part of the pretest questionnaire regarding the questionnaire itself.}
\resizebox{\textwidth}{!}{%
\begin{tabular}{|l|l|}
\hline
No. & Question \\ \hline
1 & Do you think that the questions from this questionnaire were easy to understand? If not, why? \\ \hline
2 & Are there repeated questions in this questionnaire? If yes, which? \\ \hline
3 & Would you include other questions on this questionnaire? If yes, which? \\ \hline
4 & Would you change the text of one or more questions in this questionnaire? If yes, which? \\ \hline
5 & \begin{tabular}[c]{@{}l@{}}Do you think a question comparing the generated maps to real cave representation is necessary \\ for this questionnaire?\end{tabular} \\ \hline
\end{tabular}%
}
\legend{Source: Table provided by author.}
\label{table:pretest_quest}
\end{table}

\chapter{Second version of the questionnaire}
\label{appendix:c}

% Please add the following required packages to your document preamble:
% \usepackage{graphicx}
\begin{table}[h]
\caption{Second version of the questionnaire.}
\resizebox{\textwidth}{!}{%
\begin{tabular}{|l|l|l|}
\hline
No. & Question & Justification \\ \hline
1 & \begin{tabular}[c]{@{}l@{}}There was something interesting in the cave \\ maps that caught my attention.\end{tabular} & \begin{tabular}[c]{@{}l@{}}Question related to the Attention component of the \\ ARCS model.\end{tabular} \\ \hline
2 & \begin{tabular}[c]{@{}l@{}}The structure of the caves was harder to \\ comprehend than I would have liked them to be.\end{tabular} & \begin{tabular}[c]{@{}l@{}}Question related to the Confidence component of\\ the ARCS model.\end{tabular} \\ \hline
3 & \begin{tabular}[c]{@{}l@{}}I feel like it would be easy for me to get lost in\\ more than one of the observed maps.\end{tabular} & \begin{tabular}[c]{@{}l@{}}Question related to the criteria: maps should have\\ an entrance, an exit, and a path between them.\end{tabular} \\ \hline
4 & \begin{tabular}[c]{@{}l@{}}In more than one of the generated maps, the \\ structure of the map looked natural, it didn't \\ look like it was generated by an algorithm.\end{tabular} & \begin{tabular}[c]{@{}l@{}}Question related to the criteria: generated maps\\ should have a natural look.\end{tabular} \\ \hline
5 & \begin{tabular}[c]{@{}l@{}}While I observed the maps, I felt like I was\\ seeing a representation of real caves.\end{tabular} & \begin{tabular}[c]{@{}l@{}}Question related to the criteria: generated maps\\ should have a natural look.\end{tabular} \\ \hline
6 & \begin{tabular}[c]{@{}l@{}}In the generated maps, what reminds me of a \\ real cave are the used colors and textures.\end{tabular} & \begin{tabular}[c]{@{}l@{}}This question serves to evaluate how important\\ the used colors and textures are in the creation \\ of a natural look.\end{tabular} \\ \hline
7 & \begin{tabular}[c]{@{}l@{}}In the generated maps, what reminds me of a\\ real cave is the structure of paths.\end{tabular} & \begin{tabular}[c]{@{}l@{}}This question serves to evaluate how important\\ the structure of paths are in the creation of a \\ natural look.\end{tabular} \\ \hline
8 & \begin{tabular}[c]{@{}l@{}}The design of more than one map made it \\ difficult for me to keep my attention.\end{tabular} & \begin{tabular}[c]{@{}l@{}}Question related to the Attention component of \\ the ARCS model.\end{tabular} \\ \hline
9 & \begin{tabular}[c]{@{}l@{}}While looking at the maps, I feel like exploring\\ them.\end{tabular} & \begin{tabular}[c]{@{}l@{}}Question related to the criteria: maps should\\ encourage players to explore.\end{tabular} \\ \hline
10 & \begin{tabular}[c]{@{}l@{}}The design of the generated cave maps is \\ attractive.\end{tabular} & \begin{tabular}[c]{@{}l@{}}Question related to the Satisfaction component of\\ the ARCS model.\end{tabular} \\ \hline
11 & \begin{tabular}[c]{@{}l@{}}The generated structures look very similar,\\ with little variety.\end{tabular} & \begin{tabular}[c]{@{}l@{}}Question related to the criteria: maps should\\ look different from one another\end{tabular} \\ \hline
12 & \begin{tabular}[c]{@{}l@{}}I would play a game that utilizes the generated\\ maps.\end{tabular} & \begin{tabular}[c]{@{}l@{}}Question related to the Relevance component of\\ the ARCS model.\end{tabular} \\ \hline
\end{tabular}%
}
\legend{Source: Table provided by author.}
\label{table:final_quest}
\end{table}

\end{document}
