% palavras-chave
% iniciar todas com letras minúsculas, exceto no caso de abreviaturas

% Keywords precisam estar definidos ANTES do \maketitle

\begin{abstract}

Procedural Content Generation (PCG) has been extensively used in game design. With it, game content can be created automatically with limited or indirect user input. This application of PCG is important in lowering the cost and time-consumption of game design, specially in current years since the video-game industry has been experiencing a dramatic growth over the last decades. The objective of this work is to create a system that generates cave-like maps for 2D top-down games, which can serve as a blueprint for game designers to build upon on the future. Additionally, the work intends to evaluate if the generated maps have a set of qualities to make them be perceived as good or desirable by players. The system was developed in the Unity Game Engine, utilizing the C\# language and a combination of algorithms, that are presented and explained in this work. In order to evaluate the maps, criteria on what makes a good map were researched, and then applied a survey to test if the generated maps satisfied said criteria. A pretest questionnaire was made and answered by 23 participants, its results were used to develop an improved version of it, which was then answered by 163 participants. Finally, 9 out of 12 questions of the survey have reached their desired result.
\end{abstract}

\begin{englishabstract}
{Geração procedural de mapas de cavernas para jogos 2D top-down}
{Jogo eletrônico, geração procedural, calabouço, Unity, caverna}

A Geração Procedural de Conteúdo (PCG) tem sido amplamente utilizada no design de jogos eletrônicos. Com sua utilização, o conteúdo de um jogo pode ser criado automaticamente, através do uso de entradas de usuário limitadas ou indiretas. Esta aplicação de PCG faz-se importante para reduzir o custo e o consumo de tempo do design de jogos, especialmente nos anos atuais, já que a indústria de videogames sofreu um crescimento significativo nas últimas décadas. O objetivo deste trabalho é criar um sistema que gere mapas semelhantes a cavernas para jogos 2D top-down, que podem servir como uma base de desenvolvimento para designers de jogos. Além disso, o trabalho pretende avaliar se os mapas gerados possuem um conjunto de qualidades para que sejam percebidos como bons ou desejáveis pelos jogadores. O sistema foi desenvolvido utilizando-se do motor de jogos Unity, da linguagem C\# e de uma combinação de algoritmos, que são apresentados e explicados neste trabalho. Para avaliar os mapas, pesquisou-se por critérios sobre o que constitui um bom mapa e, em seguida, uma pesquisa foi realizada através de um questionário para testar se os mapas gerados satisfazem esses critérios. Um questionário de pré-teste foi aplicado e respondido por 23 participantes, seus resultados foram utilizados no desenvolvimento de uma versão aperfeiçoada deste, que foi respondida por 163 participantes. Por fim, 9 das 12 questões alcançaram seus resultados desejados.

\end{englishabstract}



